
%% bare_jrnl.tex
%% V1.4b
%% 2015/08/26
%% by Michael Shell
%% see http://www.michaelshell.org/
%% for current contact information.
%%
%% This is a skeleton file demonstrating the use of IEEEtran.cls
%% (requires IEEEtran.cls version 1.8b or later) with an IEEE
%% journal paper.
%%
%% Support sites:
%% http://www.michaelshell.org/tex/ieeetran/
%% http://www.ctan.org/pkg/ieeetran
%% and
%% http://www.ieee.org/

%%*************************************************************************
%% Legal Notice:
%% This code is offered as-is without any warranty either expressed or
%% implied; without even the implied warranty of MERCHANTABILITY or
%% FITNESS FOR A PARTICULAR PURPOSE!
%% User assumes all risk.
%% In no event shall the IEEE or any contributor to this code be liable for
%% any damages or losses, including, but not limited to, incidental,
%% consequential, or any other damages, resulting from the use or misuse
%% of any information contained here.
%%
%% All comments are the opinions of their respective authors and are not
%% necessarily endorsed by the IEEE.
%%
%% This work is distributed under the LaTeX Project Public License (LPPL)
%% ( http://www.latex-project.org/ ) version 1.3, and may be freely used,
%% distributed and modified. A copy of the LPPL, version 1.3, is included
%% in the base LaTeX documentation of all distributions of LaTeX released
%% 2003/12/01 or later.
%% Retain all contribution notices and credits.
%% ** Modified files should be clearly indicated as such, including  **
%% ** renaming them and changing author support contact information. **
%%*************************************************************************


% *** Authors should verify (and, if needed, correct) their LaTeX system  ***
% *** with the testflow diagnostic prior to trusting their LaTeX platform ***
% *** with production work. The IEEE's font choices and paper sizes can   ***
% *** trigger bugs that do not appear when using other class files.       ***                          ***
% The testflow support page is at:
% http://www.michaelshell.org/tex/testflow/



\documentclass[journal]{IEEEtran}

%\usepackage{algorithm}
%\usepackage{algorithmicx}
%\usepackage{algpseudocode}
%\usepackage{amsmath}

%\renewcommand{\algorithmicrequire}{\textbf{Input:}}  % Use Input in the format of Algorithm
%\renewcommand{\algorithmicensure}{\textbf{Output:}} % Use Output in the format of Algorithm
%
% If IEEEtran.cls has not been installed into the LaTeX system files,
% manually specify the path to it like:
% \documentclass[journal]{../sty/IEEEtran}


\usepackage{latexsym}
\usepackage{amsmath}
\usepackage{amssymb}
\usepackage{graphicx}
\usepackage{subfigure}

\makeatletter
\newif\if@restonecol
\makeatother
\let\algorithm\relax
\let\endalgorithm\relax
\usepackage[linesnumbered,ruled,vlined]{algorithm2e}%[ruled,vlined]{
\usepackage{algpseudocode}
\usepackage{amsmath}

\renewcommand{\algorithmicrequire}{\textbf{Input:}}  % Use Input in the format of Algorithm
\renewcommand{\algorithmicensure}{\textbf{Output:}} % Use Output in the format of Algorithm


% Some very useful LaTeX packages include:
% (uncomment the ones you want to load)


% *** MISC UTILITY PACKAGES ***
%
%\usepackage{ifpdf}
% Heiko Oberdiek's ifpdf.sty is very useful if you need conditional
% compilation based on whether the output is pdf or dvi.
% usage:
% \ifpdf
%   % pdf code
% \else
%   % dvi code
% \fi
% The latest version of ifpdf.sty can be obtained from:
% http://www.ctan.org/pkg/ifpdf
% Also, note that IEEEtran.cls V1.7 and later provides a builtin
% \ifCLASSINFOpdf conditional that works the same way.
% When switching from latex to pdflatex and vice-versa, the compiler may
% have to be run twice to clear warning/error messages.






% *** CITATION PACKAGES ***
%
%\usepackage{cite}
% cite.sty was written by Donald Arseneau
% V1.6 and later of IEEEtran pre-defines the format of the cite.sty package
% \cite{} output to follow that of the IEEE. Loading the cite package will
% result in citation numbers being automatically sorted and properly
% "compressed/ranged". e.g., [1], [9], [2], [7], [5], [6] without using
% cite.sty will become [1], [2], [5]--[7], [9] using cite.sty. cite.sty's
% \cite will automatically add leading space, if needed. Use cite.sty's
% noadjust option (cite.sty V3.8 and later) if you want to turn this off
% such as if a citation ever needs to be enclosed in parenthesis.
% cite.sty is already installed on most LaTeX systems. Be sure and use
% version 5.0 (2009-03-20) and later if using hyperref.sty.
% The latest version can be obtained at:
% http://www.ctan.org/pkg/cite
% The documentation is contained in the cite.sty file itself.






% *** GRAPHICS RELATED PACKAGES ***
%
\ifCLASSINFOpdf
  % \usepackage[pdftex]{graphicx}
  % declare the path(s) where your graphic files are
  % \graphicspath{{../pdf/}{../jpeg/}}
  % and their extensions so you won't have to specify these with
  % every instance of \includegraphics
  % \DeclareGraphicsExtensions{.pdf,.jpeg,.png}
\else
  % or other class option (dvipsone, dvipdf, if not using dvips). graphicx
  % will default to the driver specified in the system graphics.cfg if no
  % driver is specified.
  % \usepackage[dvips]{graphicx}
  % declare the path(s) where your graphic files are
  % \graphicspath{{../eps/}}
  % and their extensions so you won't have to specify these with
  % every instance of \includegraphics
  % \DeclareGraphicsExtensions{.eps}
\fi
% graphicx was written by David Carlisle and Sebastian Rahtz. It is
% required if you want graphics, photos, etc. graphicx.sty is already
% installed on most LaTeX systems. The latest version and documentation
% can be obtained at:
% http://www.ctan.org/pkg/graphicx
% Another good source of documentation is "Using Imported Graphics in
% LaTeX2e" by Keith Reckdahl which can be found at:
% http://www.ctan.org/pkg/epslatex
%
% latex, and pdflatex in dvi mode, support graphics in encapsulated
% postscript (.eps) format. pdflatex in pdf mode supports graphics
% in .pdf, .jpeg, .png and .mps (metapost) formats. Users should ensure
% that all non-photo figures use a vector format (.eps, .pdf, .mps) and
% not a bitmapped formats (.jpeg, .png). The IEEE frowns on bitmapped formats
% which can result in "jaggedy"/blurry rendering of lines and letters as
% well as large increases in file sizes.
%
% You can find documentation about the pdfTeX application at:
% http://www.tug.org/applications/pdftex





% *** MATH PACKAGES ***
%
%\usepackage{amsmath}
% A popular package from the American Mathematical Society that provides
% many useful and powerful commands for dealing with mathematics.
%
% Note that the amsmath package sets \interdisplaylinepenalty to 10000
% thus preventing page breaks from occurring within multiline equations. Use:
%\interdisplaylinepenalty=2500
% after loading amsmath to restore such page breaks as IEEEtran.cls normally
% does. amsmath.sty is already installed on most LaTeX systems. The latest
% version and documentation can be obtained at:
% http://www.ctan.org/pkg/amsmath





% *** SPECIALIZED LIST PACKAGES ***
%
%\usepackage{algorithmic}
% algorithmic.sty was written by Peter Williams and Rogerio Brito.
% This package provides an algorithmic environment fo describing algorithms.
% You can use the algorithmic environment in-text or within a figure
% environment to provide for a floating algorithm. Do NOT use the algorithm
% floating environment provided by algorithm.sty (by the same authors) or
% algorithm2e.sty (by Christophe Fiorio) as the IEEE does not use dedicated
% algorithm float types and packages that provide these will not provide
% correct IEEE style captions. The latest version and documentation of
% algorithmic.sty can be obtained at:
% http://www.ctan.org/pkg/algorithms
% Also of interest may be the (relatively newer and more customizable)
% algorithmicx.sty package by Szasz Janos:
% http://www.ctan.org/pkg/algorithmicx




% *** ALIGNMENT PACKAGES ***
%
%\usepackage{array}
% Frank Mittelbach's and David Carlisle's array.sty patches and improves
% the standard LaTeX2e array and tabular environments to provide better
% appearance and additional user controls. As the default LaTeX2e table
% generation code is lacking to the point of almost being broken with
% respect to the quality of the end results, all users are strongly
% advised to use an enhanced (at the very least that provided by array.sty)
% set of table tools. array.sty is already installed on most systems. The
% latest version and documentation can be obtained at:
% http://www.ctan.org/pkg/array


% IEEEtran contains the IEEEeqnarray family of commands that can be used to
% generate multiline equations as well as matrices, tables, etc., of high
% quality.




% *** SUBFIGURE PACKAGES ***
%\ifCLASSOPTIONcompsoc
%  \usepackage[caption=false,font=normalsize,labelfont=sf,textfont=sf]{subfig}
%\else
%  \usepackage[caption=false,font=footnotesize]{subfig}
%\fi
% subfig.sty, written by Steven Douglas Cochran, is the modern replacement
% for subfigure.sty, the latter of which is no longer maintained and is
% incompatible with some LaTeX packages including fixltx2e. However,
% subfig.sty requires and automatically loads Axel Sommerfeldt's caption.sty
% which will override IEEEtran.cls' handling of captions and this will result
% in non-IEEE style figure/table captions. To prevent this problem, be sure
% and invoke subfig.sty's "caption=false" package option (available since
% subfig.sty version 1.3, 2005/06/28) as this is will preserve IEEEtran.cls
% handling of captions.
% Note that the Computer Society format requires a larger sans serif font
% than the serif footnote size font used in traditional IEEE formatting
% and thus the need to invoke different subfig.sty package options depending
% on whether compsoc mode has been enabled.
%
% The latest version and documentation of subfig.sty can be obtained at:
% http://www.ctan.org/pkg/subfig




% *** FLOAT PACKAGES ***
%
%\usepackage{fixltx2e}
% fixltx2e, the successor to the earlier fix2col.sty, was written by
% Frank Mittelbach and David Carlisle. This package corrects a few problems
% in the LaTeX2e kernel, the most notable of which is that in current
% LaTeX2e releases, the ordering of single and double column floats is not
% guaranteed to be preserved. Thus, an unpatched LaTeX2e can allow a
% single column figure to be placed prior to an earlier double column
% figure.
% Be aware that LaTeX2e kernels dated 2015 and later have fixltx2e.sty's
% corrections already built into the system in which case a warning will
% be issued if an attempt is made to load fixltx2e.sty as it is no longer
% needed.
% The latest version and documentation can be found at:
% http://www.ctan.org/pkg/fixltx2e


%\usepackage{stfloats}
% stfloats.sty was written by Sigitas Tolusis. This package gives LaTeX2e
% the ability to do double column floats at the bottom of the page as well
% as the top. (e.g., "\begin{figure*}[!b]" is not normally possible in
% LaTeX2e). It also provides a command:
%\fnbelowfloat
% to enable the placement of footnotes below bottom floats (the standard
% LaTeX2e kernel puts them above bottom floats). This is an invasive package
% which rewrites many portions of the LaTeX2e float routines. It may not work
% with other packages that modify the LaTeX2e float routines. The latest
% version and documentation can be obtained at:
% http://www.ctan.org/pkg/stfloats
% Do not use the stfloats baselinefloat ability as the IEEE does not allow
% \baselineskip to stretch. Authors submitting work to the IEEE should note
% that the IEEE rarely uses double column equations and that authors should try
% to avoid such use. Do not be tempted to use the cuted.sty or midfloat.sty
% packages (also by Sigitas Tolusis) as the IEEE does not format its papers in
% such ways.
% Do not attempt to use stfloats with fixltx2e as they are incompatible.
% Instead, use Morten Hogholm'a dblfloatfix which combines the features
% of both fixltx2e and stfloats:
%
% \usepackage{dblfloatfix}
% The latest version can be found at:
% http://www.ctan.org/pkg/dblfloatfix




%\ifCLASSOPTIONcaptionsoff
%  \usepackage[nomarkers]{endfloat}
% \let\MYoriglatexcaption\caption
% \renewcommand{\caption}[2][\relax]{\MYoriglatexcaption[#2]{#2}}
%\fi
% endfloat.sty was written by James Darrell McCauley, Jeff Goldberg and
% Axel Sommerfeldt. This package may be useful when used in conjunction with
% IEEEtran.cls'  captionsoff option. Some IEEE journals/societies require that
% submissions have lists of figures/tables at the end of the paper and that
% figures/tables without any captions are placed on a page by themselves at
% the end of the document. If needed, the draftcls IEEEtran class option or
% \CLASSINPUTbaselinestretch interface can be used to increase the line
% spacing as well. Be sure and use the nomarkers option of endfloat to
% prevent endfloat from "marking" where the figures would have been placed
% in the text. The two hack lines of code above are a slight modification of
% that suggested by in the endfloat docs (section 8.4.1) to ensure that
% the full captions always appear in the list of figures/tables - even if
% the user used the short optional argument of \caption[]{}.
% IEEE papers do not typically make use of \caption[]'s optional argument,
% so this should not be an issue. A similar trick can be used to disable
% captions of packages such as subfig.sty that lack options to turn off
% the subcaptions:
% For subfig.sty:
% \let\MYorigsubfloat\subfloat
% \renewcommand{\subfloat}[2][\relax]{\MYorigsubfloat[]{#2}}
% However, the above trick will not work if both optional arguments of
% the \subfloat command are used. Furthermore, there needs to be a
% description of each subfigure *somewhere* and endfloat does not add
% subfigure captions to its list of figures. Thus, the best approach is to
% avoid the use of subfigure captions (many IEEE journals avoid them anyway)
% and instead reference/explain all the subfigures within the main caption.
% The latest version of endfloat.sty and its documentation can obtained at:
% http://www.ctan.org/pkg/endfloat
%
% The IEEEtran \ifCLASSOPTIONcaptionsoff conditional can also be used
% later in the document, say, to conditionally put the References on a
% page by themselves.




% *** PDF, URL AND HYPERLINK PACKAGES ***
%
%\usepackage{url}
% url.sty was written by Donald Arseneau. It provides better support for
% handling and breaking URLs. url.sty is already installed on most LaTeX
% systems. The latest version and documentation can be obtained at:
% http://www.ctan.org/pkg/url
% Basically, \url{my_url_here}.




% *** Do not adjust lengths that control margins, column widths, etc. ***
% *** Do not use packages that alter fonts (such as pslatex).         ***
% There should be no need to do such things with IEEEtran.cls V1.6 and later.
% (Unless specifically asked to do so by the journal or conference you plan
% to submit to, of course. )


% correct bad hyphenation here
\hyphenation{op-tical net-works semi-conduc-tor}


\begin{document}
\bibliographystyle{unsrt}
%
% paper title
% Titles are generally capitalized except for words such as a, an, and, as,
% at, but, by, for, in, nor, of, on, or, the, to and up, which are usually
% not capitalized unless they are the first or last word of the title.
% Linebreaks \\ can be used within to get better formatting as desired.
% Do not put math or special symbols in the title.
\title{Decentralized Motion Planning for Multi-Agent System with Tight Coupled Tasks \\under LTL Specifications}
%
%
% author names and IEEE memberships
% note positions of commas and nonbreaking spaces ( ~ ) LaTeX will not break
% a structure at a ~ so this keeps an author's name from being broken across
% two lines.
% use \thanks{} to gain access to the first footnote area
% a separate \thanks must be used for each paragraph as LaTeX2e's \thanks
% was not built to handle multiple paragraphs
%

\author{Michael~Shell,~\IEEEmembership{Member,~IEEE,}
        John~Doe,~\IEEEmembership{Fellow,~OSA,}
        and~Jane~Doe,~\IEEEmembership{Life~Fellow,~IEEE}% <-this % stops a space
\thanks{M. Shell was with the Department
of Electrical and Computer Engineering, Georgia Institute of Technology, Atlanta,
GA, 30332 USA e-mail: (see http://www.michaelshell.org/contact.html).}% <-this % stops a space
\thanks{J. Doe and J. Doe are with Anonymous University.}% <-this % stops a space
\thanks{Manuscript received April 19, 2005; revised August 26, 2015.}}

% note the % following the last \IEEEmembership and also \thanks -
% these prevent an unwanted space from occurring between the last author name
% and the end of the author line. i.e., if you had this:
%
% \author{....lastname \thanks{...} \thanks{...} }
%                     ^------------^------------^----Do not want these spaces!
%
% a space would be appended to the last name and could cause every name on that
% line to be shifted left slightly. This is one of those "LaTeX things". For
% instance, "\textbf{A} \textbf{B}" will typeset as "A B" not "AB". To get
% "AB" then you have to do: "\textbf{A}\textbf{B}"
% \thanks is no different in this regard, so shield the last } of each \thanks
% that ends a line with a % and do not let a space in before the next \thanks.
% Spaces after \IEEEmembership other than the last one are OK (and needed) as
% you are supposed to have spaces between the names. For what it is worth,
% this is a minor point as most people would not even notice if the said evil
% space somehow managed to creep in.



% The paper headers
\markboth{Journal of \LaTeX\ Class Files,~Vol.~14, No.~8, August~2015}%
{Shell \MakeLowercase{\textit{et al.}}: Bare Demo of IEEEtran.cls for IEEE Journals}
% The only time the second header will appear is for the odd numbered pages
% after the title page when using the twoside option.
%
% *** Note that you probably will NOT want to include the author's ***
% *** name in the headers of peer review papers.                   ***
% You can use \ifCLASSOPTIONpeerreview for conditional compilation here if
% you desire.




% If you want to put a publisher's ID mark on the page you can do it like
% this:
%\IEEEpubid{0000--0000/00\$00.00~\copyright~2015 IEEE}
% Remember, if you use this you must call \IEEEpubidadjcol in the second
% column for its text to clear the IEEEpubid mark.



% use for special paper notices
%\IEEEspecialpapernotice{(Invited Paper)}




% make the title area
\maketitle

% As a general rule, do not put math, special symbols or citations
% in the abstract or keywords.
\begin{abstract}
This paper considers the decentralized motion planning under tight coupled task specifications of multi-agent systems. Multi-agent systems consist of heterogeneous groups of homogeneous agents. The assigned tasks are specified as Linear Temporal Logic (LTL) formulae. A decentralized motion planning scheme is proposed to alleviate the massive computational complexity of the centralized multi-agent motion planning, where couple-edges are proposed to eliminate the coupling among tight tasks, and the online collaboration scheme based on the real-time communication is employed to ensure the satisfaction of task specifications without the central collaboration. In some scenarios such as the task swapping and inheriting, this scheme can't synthesize the initial motion paths because only local agent model is considered. Then the union agent model is proposed to plan the motion paths and reduce the computational complexity in these scenarios. The overall scheme is demonstrated by a simulation in a warehouse scenario and 14 agents are assigned tight coupled task specifications.
\end{abstract}

% Note that keywords are not normally used for peerreview papers.
\begin{IEEEkeywords}
multi-agent system, linear temporal logics (LTLs), decentralized motion planning.
\end{IEEEkeywords}






% For peer review papers, you can put extra information on the cover
% page as needed:
% \ifCLASSOPTIONpeerreview
% \begin{center} \bfseries EDICS Category: 3-BBND \end{center}
% \fi
%
% For peerreview papers, this IEEEtran command inserts a page break and
% creates the second title. It will be ignored for other modes.
\IEEEpeerreviewmaketitle



\section{Introduction}
% The very first letter is a 2 line initial drop letter followed
% by the rest of the first word in caps.
%
% form to use if the first word consists of a single letter:
% \IEEEPARstart{A}{demo} file is ....
%
% form to use if you need the single drop letter followed by
% normal text (unknown if ever used by the IEEE):
% \IEEEPARstart{A}{}demo file is ....
%
% Some journals put the first two words in caps:
% \IEEEPARstart{T}{his demo} file is ....
%
% Here we have the typical use of a "T" for an initial drop letter
% and "HIS" in caps to complete the first word.
\IEEEPARstart{T}{emporal} logic based motion planning has gained attention significantly over the last decade. The temporal logic, such as the linear temporal logic (LTL) has usage as offering formal languages which specify the tasks in a high level. The formal languages are cast in terms of graph models \cite{GASTIN2001Fast}, and graph search algorithms are applied to find the discrete motion paths, which are implemented through the low-level physical systems.\par
LTL has been employed by multi-agent systems to plan the motion paths under complex task specifications. LTL is able to model the complex logic. The complete solution space can be spanned through the specifications. Meanwhile, the specifications can be relaxed when the deadlock occurs \cite{guo2013reconfiguration}. In \cite{Chen2013A} and \cite{chen2012formal}, the essential LTL is employed by a multi-agent system to collaborate in the Robotic Urban-Like Environment(RULE), which establishes the scheme of the application of LTL in the static workspace. With respect to the dynamic workspace, agents need to revise their motion paths, \cite{guo2013revising} and \cite{guo2015multi} propose a motion path revision method based on knowledge transferring and real time DijksTA algorithm. If the LTL specifications were infeasible, \cite{guo2013reconfiguration} relaxes the infeasible specifications and synthesizes the motion paths that fulfill the infeasible specifications most. \par
A key challenge of applying LTL to the motion planning of multi-agent systems is the massive computational complexity, which is caused by the production operation. In centralized motion planning method, each agent first needs to synthesize its local product automaton, whose complexity exponential increases with the number of states. Then all the agents need to synthesize the final agent model through the production of all the local product automata.

%Due to the complex environment and tasks, the product automaton of each agent is already very complicated. In order to reduce the computational complexity in application, the motion planning scheme needs to be decentralized.

To alleviate the massive amount of calculation, many researches have been carried out. One side, the horizon of each agent is receded and the goal generator is proposed to preserve the desired temporal properties in each horizon segment \cite{wongpiromsarn2012receding},\cite{tumova2014receding}. In addition, the reduction of computational complexity through the decomposition of LTL is studied widely. \cite{schillinger2018decomposition} demonstrates an automata-based approach to decompose LTL specifications into sets of independently executable task specifications and the hierarchical decomposition of LTL is introduced in \cite{meyer2017hierarchical}. Recently, a framework using Tableau approach to decompose input LTL task specifications is proposed in \cite{al2018efficient}. Compared to the decomposition of LTL specifications, the decentralization of motion planning tends to reduce the computational complexity more efficiently. In \cite{tumova2015decomposition} the planning procedure is decentralized through a two-phase automata-based solution, where distributed task specifications are employed. Then distributed task specifications are further applied in \cite{guo2014cooperative} where the connectivity constraints are imposed on the multi-agent systems. With the preliminary decentralization methods, the reconfiguration of distributed plan is studied in \cite{guo2014distributed} to adapt the decentralization methods to the dynamical environment. Most of previous frameworks depend on the local LTL task specifications and can't handle coupled tasks. The coupled tasks require the collaboration of agents, such as agent A needs the assistance of the agent B to deliver the cargoes. In this regard, \cite{guo2017task} presents a product-free framework towards the loosely coupled task, which defines the dependencies among motions of different agents. Each agent synthesizes an initial motion path through its local product automaton. However, with respect to the tight tasks, such as one agent can't cross some region until the other agent arrives at the a specific region, due to the association between the task and the motions of multiple agents, the initial motion paths can't be synthesized through the local product automaton, which may cause the failure of the collaboration. Meanwhile, towards complex tight tasks, the dependencies among motions can't be defined clearly. In addition, the task swapping method in \cite{guo2017task} may cause the failure of task collaboration due to the requirement of collaboration integrity under the tight coupled tasks.

To reduce the computational complexity of motion planning under tight coupled tasks specifications, a decentralized motion planning method is proposed in this paper. At first the tight coupled task is decoupled, then the communication model in \cite{guo2017task} is adapted to guarantee the abidance of tasks during the motion paths are executed. In the scenarios that the motion planning is associated with the models of two or more agents, such as the task swapping and inheriting, the union agent model is proposed to reduce the computational complexity. Our contribution can be summarized as follows: (1) A formal definition of tight coupled task specifications is introduced and the decentralized motion planning scheme based on the decoupling of the tight task and real-time communication is proposed to reduce the computational complexity. (2) Moreover, Towards the scenarios such as task swapping and inheriting the union agent model is designed to reduce the computational complexity.

The remaining of this paper is organized as follows: Some preliminaries are introduced in Sec.\uppercase\expandafter{\romannumeral2}. Sec.\uppercase\expandafter{\romannumeral3} states formally the problem. Sec.\uppercase\expandafter{\romannumeral4} presents the decentralized synthesis of the initial motion path of each agent. The decentralized collaboration of agents is described in Sec.\uppercase\expandafter{\romannumeral5}. The union agent model is demonstrated in Sec.\uppercase\expandafter{\romannumeral6}. Sec.\uppercase\expandafter{\romannumeral7} shows the overall structure of our framework and case studies are shown in Sec.\uppercase\expandafter{\romannumeral8}. Finally, a summary and future work are given in Sec.\uppercase\expandafter{\romannumeral9}.
% You must have at least 2 lines in the paragraph with the drop letter
% (should never be an issue)


%\subsection{Subsection Heading Here}
%Subsection text here.

% needed in second column of first page if using \IEEEpubid
%\IEEEpubidadjcol

%\subsubsection{Subsubsection Heading Here}
%Subsubsection text here.


% An example of a floating figure using the graphicx package.
% Note that \label must occur AFTER (or within) \caption.
% For figures, \caption should occur after the \includegraphics.
% Note that IEEEtran v1.7 and later has special internal code that
% is designed to preserve the operation of \label within \caption
% even when the captionsoff option is in effect. However, because
% of issues like this, it may be the safest practice to put all your
% \label just after \caption rather than within \caption{}.
%
% Reminder: the "draftcls" or "draftclsnofoot", not "draft", class
% option should be used if it is desired that the figures are to be
% displayed while in draft mode.
%
%\begin{figure}[!t]
%\centering
%\includegraphics[width=2.5in]{myfigure}
% where an .eps filename suffix will be assumed under latex,
% and a .pdf suffix will be assumed for pdflatex; or what has been declared
% via \DeclareGraphicsExtensions.
%\caption{Simulation results for the network.}
%\label{fig_sim}
%\end{figure}

% Note that the IEEE typically puts floats only at the top, even when this
% results in a large percentage of a column being occupied by floats.


% An example of a double column floating figure using two subfigures.
% (The subfig.sty package must be loaded for this to work.)
% The subfigure \label commands are set within each subfloat command,
% and the \label for the overall figure must come after \caption.
% \hfil is used as a separator to get equal spacing.
% Watch out that the combined width of all the subfigures on a
% line do not exceed the text width or a line break will occur.
%
%\begin{figure*}[!t]
%\centering
%\subfloat[Case I]{\includegraphics[width=2.5in]{box}%
%\label{fig_first_case}}
%\hfil
%\subfloat[Case II]{\includegraphics[width=2.5in]{box}%
%\label{fig_second_case}}
%\caption{Simulation results for the network.}
%\label{fig_sim}
%\end{figure*}
%
% Note that often IEEE papers with subfigures do not employ subfigure
% captions (using the optional argument to \subfloat[]), but instead will
% reference/describe all of them (a), (b), etc., within the main caption.
% Be aware that for subfig.sty to generate the (a), (b), etc., subfigure
% labels, the optional argument to \subfloat must be present. If a
% subcaption is not desired, just leave its contents blank,
% e.g., \subfloat[].


% An example of a floating table. Note that, for IEEE style tables, the
% \caption command should come BEFORE the table and, given that table
% captions serve much like titles, are usually capitalized except for words
% such as a, an, and, as, at, but, by, for, in, nor, of, on, or, the, to
% and up, which are usually not capitalized unless they are the first or
% last word of the caption. Table text will default to \footnotesize as
% the IEEE normally uses this smaller font for tables.
% The \label must come after \caption as always.
%
%\begin{table}[!t]
%% increase table row spacing, adjust to taste
%\renewcommand{\arraystretch}{1.3}
% if using array.sty, it might be a good idea to tweak the value of
% \extrarowheight as needed to properly center the text within the cells
%\caption{An Example of a Table}
%\label{table_example}
%\centering
%% Some packages, such as MDW tools, offer better commands for making tables
%% than the plain LaTeX2e tabular which is used here.
%\begin{tabular}{|c||c|}
%\hline
%One & Two\\
%\hline
%Three & Four\\
%\hline
%\end{tabular}
%\end{table}


% Note that the IEEE does not put floats in the very first column
% - or typically anywhere on the first page for that matter. Also,
% in-text middle ("here") positioning is typically not used, but it
% is allowed and encouraged for Computer Society conferences (but
% not Computer Society journals). Most IEEE journals/conferences use
% top floats exclusively.
% Note that, LaTeX2e, unlike IEEE journals/conferences, places
% footnotes above bottom floats. This can be corrected via the
% \fnbelowfloat command of the stfloats package.
\section{Preliminaries}
\subsection{Linear Temporal Logic and B\"{u}chi Automaton}
Atomic propositions (AP) are boolean variables that can be either true or false and the
ingredients of linear temporal logic formulae are sets of atomic propositions and several
boolean and temporal operators, which are specified according to the following syntax
\cite{baier2008principles}: $\varphi\ $$:$$:$$=\top|a|\varphi_1\wedge\varphi_2| \neg \varphi|\bigcirc\varphi|\varphi_1\cup\varphi_2|\Box\varphi|\Diamond\varphi|\Longrightarrow\varphi$, where $\top\ (\emph{True})$, $a\ \in AP$, $\bigcirc$ (\emph{Next}), $\cup$ (\emph{Until}), $\Box (\emph{Always})$,  $\Diamond (\emph{Eventually})$ and $\Longrightarrow (\emph{Implicate})$. We refer the readers to [add ref] to grasp the detailed semantics of LTL. $\cup\ $, $\Longrightarrow$ and $\wedge$ are considered in this paper which generate the tight coupled tasks and let the metavariable $\oplus \in \{\cup, \Longrightarrow,\wedge\}$.
Each LTL formula $\varphi$ can be translated into a B\"{u}chi Automaton (BA) $\mathcal{B}_\varphi$. It is defined as $\mathcal{B}_\varphi=\ (Q,2^{AP},\delta,Q_0,Q_F)$, where $Q$ is the finite set of all the states of $\mathcal{B}_\varphi$, $2^{AP}$ is the set of alphabets, $\delta:\ Q\times 2^{AP}\rightarrow 2^Q$ is the transition relation and $Q_F \subseteq Q$ is the set of accepting states. There are translation algorithms \cite{GASTIN2001Fast} to obtain $\mathcal{B}_\varphi$ from the LTL formula $\varphi$.
\subsection{Finite Transition System}
The environment of agents is modeled as finite transition systems (FTS) as follows \cite{baier2008principles}:
$$\mathcal{F}=(\Pi,\rightarrow,\Pi_0,L,\digamma)$$
where $\Pi$ is the set of environment regions, $\rightarrow \subseteq \Pi \times \Pi$ is the transition relation, $\Pi_0 \subseteq \Pi$ is the initial region, $L:\Pi\rightarrow 2^{\digamma}$ is the labeling function which indicates the properties holding true in $\Pi$. A path of $\mathcal{F}$ is a sequence of regions $\pi_0 \pi_1 ...\pi_s$ where $(\pi_k,\pi_{k+1})\in \rightarrow$, $\forall k=0,1,...,L-1$.

\textbf{\emph{Remark 1:}} Action models are essentially finite transition systems, therefore the method proposed in this paper is applicable for tight coupled action models at the same time.
\subsection{Product Automaton}
Given the BA $\mathcal{B}_\varphi$ and FTS $\mathcal{F}$, the product automaton $\mathcal{C}$ can be constructed as:
$$\mathcal{C}=\mathcal{B}_\varphi\otimes \mathcal{C}=(Q^{\ast},\delta,Q^{\ast}_0,Q^{\ast}_F,W^{\ast},\tau)$$
where $Q^{\ast}=\Pi\times Q$, $q^{\ast}=<\pi,q>$, $\forall \pi \in \Pi$, $\forall q \in Q$; $(<\pi_m,q_s>,<\pi_n,q_t>)\in \delta$ if $\pi_m \rightarrow \pi_n$ and $(q_s,L(\pi_m),q_t)\in \delta$; $Q^{\ast}_0=\Pi_0\times Q_0$ is the set of initial states; $Q^{\ast}_F=\Pi\times Q_F$ is the set of accepting states; $W^{\ast}:\delta\times Q^{\ast}\rightarrow \mathbb{R}^{+}$. $W^{\ast}(<\pi_m,q_s>,<\pi_n,q_t>)$ is the distance between $\pi_m$ and $\pi_n$, where $<\pi_n,q_t>\in \delta(<\pi_m,q_s>,L(\pi_m))$. $\tau$ indicates whether $(<\pi_m,q_s>,<\pi_n,q_t>)$ is a couple edge which is going to be introduced in Sec.\uppercase\expandafter{\romannumeral4}.
\section{Problem Formulation}
A multi-agent system organized by $N\geq1$ agents is considered. The sets of agents are indexed by $\mathcal{I}=\{p_i,i=1,2,...,N\}$. The agents with different LTL task specifications are divided into different subsets. Assume there are $M$ subsets after dividing, denoted by $\mathcal{G}=\{g_j,j=1,2,...,M\}$. $\mathcal{G}$ satisfies: $g_s\subseteq \mathcal{I},\ \cup_{g_t\subseteq \mathcal{G}}g_t\ =\ \mathcal{I}$ and $g_s\cap g_t\ =\ \emptyset$, $\forall g_t,g_s \subseteq \mathcal{G}$.
Each agent only knows a part of the entire workspace. The workspace of each agent consists of $L$ partitions, denoted by $\Pi^{p_i}=\{\pi_1,\pi_2,...,\pi_L\}$. The atomic propositions in $\digamma^{p_i}$ describe the properties of $\Pi^{p_i}$.
\subsection{Communication Model}
Similar to the communication model in \cite{guo2017task}, the communication network $\mathcal{V}=(\mathcal{N},\mathcal{E}(t))$ at $t>0$, where $\mathcal{N}$ is the set of nodes and $\mathcal{E}(t)\subseteq \mathcal{N} \times \mathcal{N}$ is the set of edges, has two layers that $\mathcal{E}(t)=\mathcal{E}_1(t)\cup \mathcal{E}_2(t)$. The first layer $\mathcal{E}_1(t)={(p_i,p_j),\forall i,j\in g_t}$ and $\mathcal{E}_2(t)={(p_i,p_j),\forall i\in g_t,\forall j \in g_s, t\neq s}$. $\mathcal{E}_1(t)$ is static, so all the agents belonging to the same group $g_t$ can always communicate with each other. The second layer $\mathcal{E}_2(t)$ is dynamic, which indicates that $\mathcal{E}_2(t)$ is enabled when there are messages that need to be delivered among different subsets of agents.
\subsection{Tight Coupled Task Specification}
In this paper the tight coupled task specification is considered. Compared to the loosely coupled task specifications, tight coupled task specifications are relevant to the motions of two or more agents tightly, thus the initial motion paths can't be synthesized only according to the local product automaton as previous method \cite{guo2017task}. The tight coupled task specification takes an recursive definition as following.
\par
\textbf{\emph{Definition 1:}} Given a LTL formula $\varphi$ contains temporal operators such as $\cup$, $\Longrightarrow$ and $\wedge$. Denote the terms around each temporal operators by $t^{l}_i$ and $t^{r}_i$, respectively. If $t^{l}_i$ and $t^{r}_i$ belongs to $AP$ of different agents or one of them is a tight coupled task specification at least, then $\varphi$ is a tight coupled task specification.
\par
In other words, a tight coupled task specification should combine the motions of different agents through $\cup$, $\Longrightarrow$ and $\wedge$ so that the initial motion path can't be synthesized through local product automaton. In summary, following problem is considered.
\par
\textbf{\emph{Problem 1:}} Given the product automaton $\mathcal{C}^{p_i}$ and a serious of task specifications $\varphi$ which contain tight coupled task specifications, design a decentralized collaboration scheme such that each $p_i \subseteq \mathcal{I}$ can synthesize an initial motion path and $\varphi$ is fulfilled.

\textbf{\emph{Remark 2:}} Specified as Definition 1, the loosely coupled task specifications can be transformed into tight coupled task specifications.
\section{Decentralized Initial Motion Path Synthesis}
The tight coupled task specifications are decoupled and the new product automata are constructed in this section. The initial motion paths are synthesized through the new product automata.
\subsection{Decoupling the Tight Coupled Task}
The tight coupling exists in the task collaboration process and neither the motions nor states of agents directly participate in the construction of product automata of other agents. Given the tight coupled task $\varphi=\varphi_1 \oplus \varphi_2$, the original local product automaton $\mathcal{C}^{p_i}$ can be constructed. All the atomic propositions associated with $\varphi$ are denoted by $sym(\varphi)$. Then $sym(\varphi_1) \in Q^{p_i}|_{\mathcal{F}^{p_i}}$, and $sym(\varphi_2) \notin Q^{p_i}|_{\mathcal{F}^{p_i}} $. There exists a subset of $Q^{p_i}$ denoted by $\epsilon$ that $sym(\varphi_1) \in \epsilon |_{\mathcal{F}^{p_i}}$ and $sym(\varphi_1) \notin Q^{p_i} \setminus \epsilon |_{\mathcal{F}^{p_i}}$. Then there is no edge from $Q^{p_i} \setminus \epsilon$ to $\epsilon$ so that the initial motion path can't be synthesized through $\mathcal{C}^{p_i}$.\par
\textbf{\emph{Definition 2:}} Given a product automaton $\mathcal{C}^{p_i}$, If two nodes $q_i$, $q_j$ satisfy following properties:
\par
(1) $q_i \in \epsilon$, $q_j \in Q^{p_i} \setminus \epsilon$ or vice versa.\par
(2) $\delta(q_i,q_j)\subseteq sym(\varphi_c)$ or $\delta(q_j,q_i)\subseteq sym(\varphi_c)$.

where $\varphi_c$ denotes the tight coupled part of the task specification $\varphi$.
The edges connecting $q_i$ and $q_j$ are called the couple-edges.

Previous local product algorithm \cite{baier2008principles} which combines the FTS and B\"{u}chi Automaton of each agent is adapted to find out all the couple-edges with respect to the tight coupled task specification $\varphi_c$ and all the couple-edges are added into the original product automaton. Applying Algorithm 1 a new product automaton $\mathcal{C}^{p_i}_n$ is constructed for each agent, which contains $\mathcal{C}^{p_i}$ and all the couple-edges.\par
\begin{algorithm}
  \caption{Construct New Product Automaton $\mathcal{C}^{p_i}_n$}
  \KwIn{the local FTS $\mathcal{F}^{p_i}$, the local B\"{u}chi Automaton $\mathcal{B}^{p_i}$ and the tight coupled task specification $\varphi_c$}
  \KwOut{new local product automaton $\mathcal{C}^{p_i}_n$}
  \ForEach{$\pi_l\in \Pi^{p_i},q_m\in Q^{p_i}$}
  {
    $q^{\ast}_s=<\pi_l,q_m>\in Q^{p_i,\ast}$

    \If{$\pi_l\in \Pi_0^{p_i}\ and\ q_m\in Q_0^{p_i}$}
    {
        $q^{\ast}_s\in Q^{p_i,\ast}_0$
    }
    \If{$q_m\in Q_F$}
    {
        $q^{\ast}_s\in Q^{p_i,\ast}_F$
    }
    \ForEach{$\pi_z\in Post(\pi_l),q_n \in Post(q_m)$}
    {
        $q^{\ast}_g=<\pi_z,q_n>\in Q^{p_i,\ast}$
        $d = CheckTranB(q_m,L^{p_i}(\pi_l),q_n,\mathcal{B}^{p_i})$

        \If{$d \geqslant 0$}
        {
            $q^{\ast}_g\in \delta(q^{\ast}_s)$

            $W^{p_i,\ast}(q^{\ast}_s,q^{\ast}_g)=dist(\pi_l,\pi_z)$

            $\tau(q^{\ast}_s,q^{\ast}_g) = None$
        }
        \Else
        {
            \If{$L^{p_i}(\pi_l)\in sym(\varphi_c)$}
            {
                $q^{\ast}_g\in \delta(q^{\ast}_s)$

                $W^{p_i,\ast}(q^{\ast}_s,q^{\ast}_g)=dist(\pi_l,\pi_z)$

                $\tau(q^{\ast}_s,q^{\ast}_g)=L^{p_i}(\pi_l)$
            }
        }
    }
  }
  \Return $\mathcal{C}^{p_i}_n$
\end{algorithm}

\textbf{\emph{Lemma 1:}} Through adding all the couple-edges found out by applying Algorithm 1 to $\mathcal{C}^{p_i}$, each agent can synthesize an initial accepting motion path.\par
\textbf{\emph{proof:}} According to Algorithm 1, the nodes in $Q^{p_i} \setminus \epsilon$ are connected because these nodes are correspond to independent task specifications. Through adding couple-edges into $\mathcal{C}^{p_i}$, each node in $\epsilon$ becomes reachable and $\mathcal{C}^{p_i}_n$ has at least one minimal spanning tree. Consequently, there must exist a path from any initial node in $Q_0^{p_i}$ to any accepting node in $Q_F^{p_i}$. So an initial motion path fulfilling $\varphi_c$ can be synthesized.
\subsection{Initial Motion Path Synthesis}
Given $\mathcal{C}^{p_i}_n$ and $\varphi^{p_i}$, there exists a finite path satisfying $\varphi^{p_i}$ if and only if $\mathcal{C}^{p_i}_n$ has a finite path from an initial state to an accepting state. Let $\varrho^{p_i}=q^{p_i,\ast}_0 q^{p_i,\ast}_1 ... q^{p_i,\ast}_s$ be a finite path, where $q^{p_i,\ast}_0\in Q^{p_i,\ast}_0$, $q^{p_i,\ast}_s\in Q^{p_i,\ast}_F$ and $q^{p_i,\ast}_z\in Q^{p_i,\ast}$ and $(q^{p_i,\ast}_z,q^{p_i,\ast}_{z+1})\in \delta^{p_i}$. The $cost$ of $\varrho^{p_i}$ is denoted by $cost(\varrho^{p_i},\mathcal{C}^{p_i}_n)=\sum^{s}_{k=0}W^{p_i,\ast}(q^{p_i,\ast}_k,q^{p_i,\ast}_{k+1})$. Apply Algorithm 1 in \cite{guo2013motion} to find $\varrho^{p_i}$, which utilizes DijksTA algorithm to find the shortest accepting path.

\textbf{\emph{Lemma 2:}} The proposed decentralized path synthesis method has solutions as long as the original centralized method has solutions.\par
\textbf{\emph{Proof:}} In Algorithm 1 all the couple-edges have been added into $\mathcal{C}^{p_i}$. Then towards $\mathcal{C}^{p_i}_n$, the valid task specification becomes $\varphi \setminus \varphi_c$. Synthesize an initial motion path $\varrho^{p_i}$ through the original centralized plan synthesis method, $\varrho^{p_i}$ fulfills $\varphi$ and $\varphi \setminus \varphi_c$ must be fulfilled simultaneously. So the solution space spanned by the original centralized method are contained in the solution space generated by the decentralized method. Consequently, the proposed decentralized path synthesis method has solutions as long as the original centralized method has solutions.\par

To guarantee the collaboration, a request and reply message exchange protocol in \cite{guo2017task}, which is driven by
collaborative actions, is employed during the execution of motion paths.

\textbf{\emph{Remark 3:}} Through the decoupling of tight task and the message exchange protocol, the product operation is replaced. The computational complexity is reduced through this scheme while the collaboration is guaranteed. 
%$\varrho^{p_i}$ constructed by [zhao na ge suan fa] is composed of prefix $\varrho^{p_i}_p$ and suffix $\varrho^{p_i}_s$. In order to adapt to next algorithms, $\varrho^{p_i}$ is reconstituted as following:\par
%(1)Catenate $\varrho^{p_i}_p$ and $\varrho^{p_i}_s$ and use $\gamma$ to indicate the separation point.
%
%(2)If two state nodes $q^{p_i,\ast}_s$ and $q^{p_i,\ast}_t$ satisfy the properties in $\emph{Definition 2}$, add $\tau(q^{p_i,\ast}_s,q^{p_i,\ast}_t)$ to the attribute $req^{p_i}$ of $q^{p_i,\ast}_s$.


%\subsection{Planning In Speed Matching Horizon}
%In order to reduce the complexity of task collaboration, a receding horizon method is applied for each agent. Consider the initial reconstituted motion plan $\varrho^{p_i}$ previous, assume the current position of agent $p_i$ is $\pi^{p_i}$ which is the $l$th element of $\varrho^{p_i}$ namely $\varrho^{p_i}[l]$. Assign a receding time $T_H$ for each agent, note that different agents have different velocity so each agent will adjust its velocity according to its speed. Assume that each agent could estimate the approximate time according its current velocity, departure position and target position, then the segment the agent expected to execute in $T_H$ is denoted by $\varrho^{p_i}_H$, $\varrho^{p_i}_H = \varrho^{p_i}[l:t]$, where the index $t$ is the solution of the following optimization problem: min $t$, subject to $\sum_{k=l}^{t}T^{p_i}(\varrho^{p_i}[k],\varrho^{p_i}[k+1])\geq T_H$, this issue can be solved by iterating through the $\varrho^{p_i}$ and computing the accumulated time which is compared to $T_H$. If previous optimization problem doesn't have a solution, it means that in $T_H$ the agent can execute the rest of the suffix of $\varrho^{p_i}$, namely $\varrho^{p_i}[l:]$. Since the split point of $\varrho^{p_i}$ has been indicated previously, original optimization problem can be transformed into a new optimization problem: min $t$, subject to $\sum_{k=\gamma}^{t}T^{p_i}(\varrho^{p_i}[k],\varrho^{p_i}[k+1])\geq T_H - \sum_{k=l}^{|\varrho^{p_i}|}T^{p_i}(\varrho^{p_i}[k],\varrho^{p_i}[k+1])$, where $|\varrho^{p_i}|$ denote the length of $\varrho^{p_i}$ and $\gamma$ denote the split point index. A solution can always be found to above problem iteratively.

%\subsection{Online Collaboration Scheme}
%Through speed matching receding horizon planning, the truncation index is founded and the receding horizon motion plan $\varrho^{p_i}_H$ is synthesized. To ensure the fulfill of task specification $\varphi$, each agent needs to check whether it needs others' collaboration within $\varrho^{p_i}_H$. Since the motion plan have been reconstituted of each agent, the first request is the first item in $\varrho^{p_i}_H$ which has $req^{p_i}$ attribute.  After one agent confirms that it needs others' collaboration, a $\textbf{Request}^{p_i}$ is sent to others. More specifically, for the first item $<q_c^{p_i},req^{p_i}>\ \in \varrho^{p_i}_H$ satisfying $req^{p_i} \neq \emptyset$, where $c$ denote the index of $<\pi_c^{p_i},q_c^{p_i},req^{p_i}>$ in $\varrho^{p_i}_H$. Agent $p_i$ needs to broadcast the request message to $\mathcal{G}\setminus g_s, p_i \in g_s$, through its communicate network. $\textbf{Request}^{p_i}$ has the following format:
%$$\textbf{Request}^{p_i} = (\chi^{p_i},\pi_c^{p_i},q_c^{p_i},T^{p_i}_e)$$
%where $\chi^{p_i}$ denotes all the items in $req^{p_i}$, which indicates the collaborative motion agent $p_i$ needs in $\pi_c^{p_i}$, and $T^{p_i}_e$ denotes the estimated time spent from $\varrho^{p_i}[l]$ to $\varrho^{p_i}[c]$, namely $T^{p_i}_e=\sum_{k=l}^{c}T^{p_i}(\varrho^{p_i}_H[k],\varrho^{p_i}_H[k+1])$. Since $\varrho^{p_i}_H[c]$ is the predecessor of the first request state in $\varrho^{p_i}_H$, $\varrho^{p_i}_H[c]$ can be always reached within finite time, so $T^{p_i}_e$ is a finite positive number.\par
%If agent $p_i$ proposes a request message and there is no reply message transitorily, all the collaborations $p_i$ requests are stored in $\mathcal{R}^{p_i}_c$, the current state $\varrho^{p_i}_r$ is stored.\par
%The algorithm finding out the first request in receding horizon motion plan $\varrho^{p_i}_H$ is as Algorithm 2.\par
%%\begin{algorithm}
%%  \caption{Plan in Horizon and Request}
%%  \KwIn{$\varrho^{p_i},\pi^{p_i},T_H,\gamma$}
%%  \KwOut{$\varrho^{p_i}_H,\textbf{Request}^{p_i}$}
%%  $\varrho^{p_i}[l]=\pi^{p_i},m=0,T_m=0,\textbf{Request}^{p_i}=\emptyset$
%%
%%  \While{$T_m<T_H$}
%%  {
%%    $m=m+1$
%%
%%    $n=m$
%%
%%    \If{$m>|\varrho^{p_i}|$}
%%    {
%%        $m=\gamma-l$
%%    }
%%
%%    $T_m=T_m+T^{p_i}(\varrho^{p_i}[l+n-1],\varrho^{p_i}[l+m])$
%%
%%    $<\pi_c^{p_i},q_c^{p_i},req^{p_i}>=\varrho^{p_i}[l+m]$
%%
%%    \If{$req^{p_i}\neq \emptyset$ and $\textbf{Request}^{p_i}=\emptyset$}
%%    {
%%        \ForAll{$\chi^{p_i}\in req^{p_i}$}
%%        {
%%            add $(\chi^{p_i},\pi_c^{p_i},q_c^{p_i},T_m)$ to $\textbf{Request}^{p_i}$
%%        }
%%    }
%%  }
%%  \If{$m>n$}
%%  {
%%    $\varrho^{p_i}_H=\varrho^{p_i}[l:l+m]$
%%  }
%%  \Else
%%  {
%%    $\varrho^{p_i}_H=\varrho^{p_i}[l:]\oplus \varrho^{p_i}[\gamma:m]$
%%  }
%%  \Return $\varrho^{p_i}_H,\textbf{Request}^{p_i}$;
%%\end{algorithm}
%Once there are request messages transmitted among agents, some agents should revise their motion plan to provide assistance. The procedure of determining which agent to respond to the request is quite similar with the algorithm proposed in [add ref], only a brief overview is presented here and we refer the readers to [add ref] to grasp more detail. In order to provide collaboration, the collaborative agent needs to revise its motion plan temporarily. Assume current state of agent $p_j$ is $\varrho^{p_j}[l]$ and next accepting state is $\varrho^{p_j}[f]$, original motion plan from $\varrho^{p_j}[l]$ to $\varrho^{p_j}[f]$ is $R^{p_j}_-=\varrho^{p_j}[l:f]$ and the motion plan after revision is denoted by $R^{p_j}_+$, the index of the position in where agent $p_j$ provides collaboration is $m$. Each agent is required to calculate the cost of providing assistance in order to decide which agent to collaborate. The $balanced\ cost$ is defined as following:
%\begin{equation}
%\label{E1}
%\begin{split}
%BalCost(&R^{p_j}_+,T^{p_i}_e,\mathcal{C}^{p_j}_n)=\\
%&|(\sum_{k=l}^{m}T^{p_j}(\varrho^{p_j}[k],\varrho^{p_j}[k+1])-T^{p_i}_e)|\\
%&+\beta_{p_j} \times (Cost(R^{p_j}_+,\mathcal{C}^{p_j}_n)-Cost(R^{p_j}_-,\mathcal{C}^{p_j}_n))
%\end{split}
%\end{equation}
%where the first part denotes the time gap between the agent which sends a request message arriving the position where needs collaboration and corresponding agent arrive the position where assistance is provided. Simply, the first part denotes the waiting time of multi-agent system, which contains both the waiting time of request agent and reply agent. The second part denotes the additional cost of $R^{p_i}_+$ compared with $R^{p_i}_-$, and $\beta_{p_j}$ is a designed parameter as relative weighting. Instead of formulating the agent choice issue as an integer programming problem,[add ref], the balanced cost of each agent is transmitted among the group $g_t$, $p_j \in g_t$, since each agent calculates its balanced cost in the same way, there must exist a consensus agent $p_{min}$ whose balanced cost is minimal and $p_{min}$ will be elected to reply to corresponding $\textbf{Request}^{p_i}$.\par
%After $p_{min}$ is elected, design an attribute of agent named $state^{p_i}$ which turns into $locked$ when agent $p_i$ is providing collaboration, otherwise $state^{p_i}$ is $normal$. Meanwhile, due to the property of tight coupled task, all the request messages that have time overlapping can be replied by a group of heterogeneous agents. So towards each request message $\textbf{Request}^{p_i}$, if there is already a group of agents ${p_i}$ whose $state^{p_i}$ are $locked$, ${p_i}$ automatically become the collaborative agents and the corresponding request message becomes invisible to other agents.\par
%Towards some tight coupled task, such as agents can't across region $r_i$ until some agents reach region $r_j$, only a group of agents is needed to reply request messages from different agents. Hence if two request messages $\textbf{Request}^{p_i}$ and $\textbf{Request}^{p_j}$ have time overlapping and there is a group of agents $\partial$ replying one request messages, then $\partial$ is selected as the reply agents for the other request message $\textbf{Request}'$ and $\textbf{Request}'$ becomes invisible to other agents.\par
%$R^{p_i}_+$ can be constructed by the $bidirectional\ Dijkstra$ algorithm which utilizes the function $bidirectional\_dijkstra()$ in $Networkx$, we refer the readers to [add ref] for more details. After the reply agent $p_r$ is elected, when $p_i$ reaches the requested replying position and provides collaboration, there is a reply message sent from the reply agent to request agent, whose format is as following:
%$$\textbf{Reply}^{p_r}=(p_r,p_i,\xi)$$
%where $p_r$ denotes the grade of replying agent and $\xi$ denotes the collaboration provided.\par
%Finally, the purpose and structure of $\textbf{Confirm}^{p_i}$ are adapted to suit the tight coupled task specification. The format of confirm message is as following:
%$$\textbf{Confirm}^{p_i}=(p_i,p_r,\hat{\xi})$$
%where $p_r$ denotes the grades of all replying agents and $\hat{\xi}$ denotes that $p_i$ has relieved collaboration requirements for action $\hat{\xi}$. It is sent when the agents which send the request message separate themselves from the area where the collaboration is required and enter next state and the confirm messages informs the replying agents to recovery their $state^{p_r}$ from $locked$ to $normal$. Denote all the request, reply and confirm messages which agent $p_i$ received by $\textbf{Request}_{p_i}^{\ast}$, $\textbf{Reply}_{p_i}^{\ast}$, $\textbf{Confirm}_{p_i}^{\ast}$, respectively. The entire algorithm to handle request, reply and confirm messages is as Algorithm 3.

%\begin{algorithm}
%  \caption{Handle Reply and Confirm messages}
%  \KwIn{$\textbf{Request}_{p_i}^{\ast}$, $\textbf{Reply}_{p_i}^{\ast}$, $\textbf{Confirm}_{p_i}^{\ast}$, $\partial$, $\mathcal{R}^{p_i}_c$, $\mathcal{C}^{p_i}_n$}
%  \KwOut{$R^{p_j}_+$}
%  $\mathcal{A}^{p_i}_s=\emptyset$
%
%  \If{$\textbf{Request}_{p_i}^{\ast} \neq \emptyset$}
%  {
%    \ForEach{$\textbf{Request}^{r} = (\chi^{r},\pi_c^{r},q_c^r,T^{r}_e) \in \textbf{Request}_{p_i}^{\ast}$}
%    {
%        \If{$\partial \neq \emptyset\ and\ p_i \notin \partial$ }
%        {
%            continue
%        }
%        \ElseIf{$(\partial \neq \emptyset\ and\ p_i \in \partial)$ or $(\partial = \emptyset\ and\ min_{g_s}BalCost = p_i)$}
%        {
%            $(R^{p_i}_+,b^{p_i}_d,t^{p_i}_d)=EvalReq(R^{p_i}_-,(\chi^{p_i},\pi_c^{p_i},q_c^{p_i},T^{p_i}_e),\varrho^{p_i}[l],\mathcal{C}^{p_i}_n)$
%
%            $state^{p_i}=locked$
%
%            add $\chi^{r}$ to $\mathcal{A}^{p_i}_s$
%
%            send $\textbf{Reply}^{p_i}=(p_i,p_r,\xi)$ to $p_r$
%        }
%    }
%  }
%  \If{$\textbf{Reply}_{p_i}^{\ast} \neq \emptyset$}
%  {
%    \ForEach{$\textbf{Reply}^{r} = (p_r,p_i,\xi) \in \textbf{Reply}_{p_i}^{\ast}$}
%    {
%        \If{$\xi \in \mathcal{R}^{p_i}_c$}
%        {
%            delete $\xi$ from $\mathcal{R}^{p_i}_c$
%        }
%        \If{$\mathcal{C}^{p_i} \in Post(\varrho^{p_i}_r)$ }
%        {
%            send $\textbf{Confirm}^{p_i}=(p_i,p_r,\hat{\xi})$ to $p_r$
%        }
%    }
%  }
%  \If{$\textbf{Confirm}_{p_i}^{\ast} \neq \emptyset$}
%  {
%    \ForEach{$\textbf{Confirm}^{r} = (p_i,p_r,\hat{\xi}) \in \textbf{Confirm}_{p_i}^{\ast}$}
%    {
%        \If{$\hat{\xi} \in \mathcal{A}^{p_i}_s$}
%        {
%            delete $\hat{\xi}$ from $\mathcal{A}^{p_i}_s$
%        }
%        \If{$\mathcal{A}^{p_i}_s=\emptyset$}
%        {
%            $state^{p_i}=normal$
%        }
%    }
%  }
%  \Return{$R^{p_j}_+$}
%\end{algorithm}
%Only if $\mathcal{R}^{p_i}_c$ is empty and $state^{p_i}$ is $normal$, agent $p_i$ can continue to execute its motion plan, otherwise $p_i$ needs to wait for the completion of collaboration.
\section{Task Inheriting Under Union Agent Model}
%Even if we drive the agents according to above framework, there will still be some issues. This section we consider these issues and construct a agent scheduling system to further revise the motion plan of agents.
%\subsection{Prevention Of Long-time Demission}
%Suppose there are many agents engaged in a same task and request messages are raised frequently, which will cause the agents responding to the first request message to be stuck in the position, where the collaboration is provided, for a long time according to above framework. To prevent long-time demission, a ultimate time $T_u$ is specified, meaning that collaboration provided by one agent can only be sustained within ultimate time. Next we will elaborate how to elect an alternative to current collaborative agent in a group of homogeneous agents and minimize the overall cost.\par
%The agent will record the time that it has spent to provide assistance once it begins to collaborate, the time is denoted by $T_{p_i}^{\ast}$. When $T_{p_i}^{\ast}>T_u$, agent $p_i$ must return to its own task, if $T_u-T_{p_i}^{\ast}<T_H$, it means agent $p_i$ needs to return within $\varrho^{p_i}_H$, then a succession signal is sent to all the homogeneous agents in group $g_s$. Redefine the $balanced\ cost$ as following:
%\begin{equation}
%\label{E2}
%\begin{split}
%BalCost(&R^{p_j}_+,T_u,T_{p_i}^{\ast},\mathcal{C}^{p_j}_n)=\\
%&|(\sum_{k=l}^{m}T^{p_j}(\varrho^{p_j}[k],\varrho^{p_j}[k+1])-(T_u-T_{p_i}^{\ast}))|\\
%&+\beta_{p_j} \times (Cost(R^{p_j}_+,\mathcal{C}^{p_j}_n)-Cost(R^{p_j}_-,\mathcal{C}^{p_j}_n))
%\end{split}
%\end{equation}
%where the meaning of notations in (\ref{E2}) are the same as the notations in (\ref{E1}). The agent $p_j$ that minimizes (\ref{E2}) in group $g_s$, $p_i,p_j \in g_s$, is elected to succeed $p_i$.\par
%Suppose the message corresponding to the most recent request is $\textbf{Request}^{p_r}=(\chi^{p_r},\pi_c^{p_r},q_c^{p_r},T^{p_r}_e)$, denote $\sum_{k=l}^{m}T^{p_j}(\varrho^{p_j}[k],\varrho^{p_j}[k+1]$ by $T_{p_j}^{\ast}$, if $T_{p_j}^{\ast}<T^{p_r}_e$, $state^{p_i}$ can be set as $normal$ and $p_i$ can go back to complete its own task, otherwise $p_i$ needs to wait for the arrival of successor. During the period between stopping collaboration and returning to its own task, agent $p_i$ doesn't accept any request messages in case of being elected as collaborative agent again.
The decentralized motion planning in normal scenarios, where each agent has the individual task specification, is revealed in last section. However, towards some special scenarios such as task inheriting and task swapping between different agents, the initial motion paths can't be synthesized through above method because the local product automaton doesn't have knowledge of the product automata of agents whose task need to be inherited or swapped with. In this section, the union agent model is proposed to reduce the computational complexity of motion planning in these scenarios.

Define a virtual order $\mathcal{O}$ which indicates the importance of the individual task that each agent undertakes. If $\mathcal{O}_{p_i}\succ \mathcal{O}_{p_j}$, when $p_i$ is breakdown, $p_j$ should revise its motion path to inherit the task of $p_i$, otherwise $p_j$ should fix on its own task.\par
When $p_i$ is broken down, a $\textbf{Warning}^{p_i}$ message is sent to other agents, the format of $\textbf{Warning}^{p_i}$ is as follows:
$$\textbf{Warning}^{p_i} = (p_i,\mathcal{C}^{p_i}_n,\aleph^{p_i})$$
which means that $p_i$ needs to broadcast its grade $p_i$, local new product automaton $\mathcal{C}^{p_i}_n$ and capacity $\aleph^{p_i}$. Then all the agents qualified for capacity $\aleph^{p_i}$, denoted by $\nu$, will respond to the $\textbf{Warning}^{p_i}$. Then the agents with the same minimal $\mathcal{O}_{p_j}$ become candidates for inheriting the task of $p_i$. In order to elect an agent to respond to the warning message and revise its motion path, following problem needs to be solved:\par
\textbf{\emph{Problem 2:}} With a little symbol abuse, denote the states in $\mathcal{C}^{p_i}_n$ by $\tilde{c}_{p_i}$, the motion path of $p_j$ after revision by $R^{p_j}_+$ and the original one by $R^{p_j}_-$, respectively. Given $\mathcal{C}^{p_i}_n$ and $\mathcal{C}^{p_j}_n$, decide the grade $\varpi$ of the agent inheriting the task of $p_i$ and construct $R^{\varpi}_+$ to minimize the total cost. \par
$\mathcal{C}^{p_i}_n$ and $\mathcal{C}^{p_j}_n$ can be combined to obtain the union agent model $\widetilde{\mathcal{U}}_{ij}$, which is defined as follows:
$$\widetilde{\mathcal{U}}_{ij}=(\widetilde{Q}^{ij},\tilde{\delta}^{ij},\widetilde{Q}^{ij}_0,\widetilde{Q}^{ij}_F,\widetilde{W}^{ij},\tilde{\tau}^{ij})$$
where $\widetilde{Q}^{ij}=Q^{p_i,\ast}\cup Q^{p_j,\ast}$; $\tilde{\delta}^{ij}=\delta^{p_i}\cup \delta^{p_j} \cup (\tilde{c}_{p_i},\tilde{c}_{p_j})$, $\tilde{c}_{p_i}$ is bordered to $\tilde{c}_{p_i}$; $\widetilde{Q}^{ij}_0=Q^{p_j,\ast}_0|_{\mathcal{F}^{p_j}}\times Q^{p_i,\ast}_0|_{\mathcal{B}^{p_i}} $; $\widetilde{Q}^{ij}_F=Q^{p_i,\ast}_F$; $\widetilde{W}^{ij}=W^{p_i,\ast}\cup W^{p_j,\ast}$ and $\tilde{\tau}^{ij}=\tau^{p_i}\cup \tau^{p_j}$.\par
Assume that part of the local FTS $\mathcal{F}^{p_i}$ and $\mathcal{F}^{p_j}$are bordered and then the construction of $\widetilde{\mathcal{U}}_{ij}$ is as follows:\par
(1)Add all the nodes and edges of $\mathcal{C}^{p_i}_n$ and $\mathcal{C}^{p_j}_n$ into $\widetilde{\mathcal{U}}_{ij}$.\par
(2)Find all the position pairs $\tilde{s}=(\tilde{\eta}_{p_i},\tilde{\eta}_{p_j})$, where $\tilde{\eta}_{p_i}$ and $\tilde{\eta}_{p_j}$ appertain to $\mathcal{F}^{p_i}$ and $\mathcal{F}^{p_j}$, respectively and they are bordered, which means agent can devolve from $\tilde{\eta}_{p_i}$ to $\tilde{\eta}_{p_j}$ directly. These position pairs are called bordered position pairs and denote the set of $\tilde{\eta}_{p_i}$ and $\tilde{\eta}_{p_j}$ by $\hat{\eta}_{p_i}$ and $\hat{\eta}_{p_j}$, respectively.\par
(3)For all states $\tilde{c}_{p_i}$ and $\tilde{c}_{p_j}$ in $\mathcal{C}^{p_i}_n$ and $\mathcal{C}^{p_j}_n$, if $\tilde{c}_{p_i}|_{\mathcal{F}^{p_i}}$ and $\tilde{c}_{p_j}|_{\mathcal{F}^{p_j}}$ is a bordered position pair and there are no violations of task specifications when transferring from $\tilde{c}_{p_i}|_{\mathcal{F}^{p_i}}$ to $\tilde{c}_{p_j}|_{\mathcal{F}^{p_j}}$, then an edge $(\tilde{c}_{p_i},\tilde{c}_{p_j},\widetilde{W}^{ij}(\tilde{c}_{p_i},\tilde{c}_{p_j}))$ is added into $\widetilde{\mathcal{U}}_{ij}$, where $\widetilde{W}^{ij}(\tilde{c}_{p_i},\tilde{c}_{p_j})$ denotes the distance between $\tilde{c}_{p_i}|_{\mathcal{F}^{p_i}}$ and $\tilde{c}_{p_j}|_{\mathcal{F}^{p_j}}$. If the transition violates the coupled task specifications and $(\tilde{c}_{p_i},\tilde{c}_{p_j})$ is a couple-edge according to \emph{Definition 2}, then add $(\tilde{c}_{p_i},\tilde{c}_{p_j},\widetilde{W}^{ij}(\tilde{c}_{p_i},\tilde{c}_{p_j}))$ as a couple-edge.\par
After the union agent model is constructed, $\varpi$ is identified and $R^{\varpi}_+$ is synthesized. Denote the current state of $p_j$ is $\tilde{c}_{p_j}^{\ast}$, $p_j$ needs to interrupt its own task so that each $\tilde{c}_{p_j}$ satisfying $\tilde{c}_{p_j}|_{\mathcal{F}^{p_j}}=\tilde{c}_{p_j}^{\ast}|_{\mathcal{F}^{p_j}}$ can be the departure point of $R^{\varpi}_+$. Therefore, once $p_j$ is assigned to inherit the task of $p_i$, all the $\tilde{c}_{p_j}|_{\mathcal{B}^{p_j}}$ are deposited and $\tilde{c}_{p_j}|_{\mathcal{F}^{p_j}}$ are composed with the $Q^{p_i,\ast}_0|_{\mathcal{B}^{p_i}}$.

\textbf{\emph{Remark 4:}} Actually, the effect of above method is the same as the union of $\mathcal{F}^{p_j}$ and $\mathcal{C}^{p_i}_n$. The $\tilde{c}_{p_j}|_{\mathcal{B}^{p_j}}$ are deposited instead of being abandoned in order to use the same model to drive $p_j$ to return to its own task after $p_i$ is repaired.

%It is obvious that the suffix $R^{\varpi,s}_+$ is the same as $R^{\varpi,s}_-$, so the $\emph{Problem 2}$ can be transformed to the synthesis of $R^{\varpi,p}_+$ from any $\tilde{c}_{p_j}$ satisfying  $\tilde{c}_{p_j}|_{\mathcal{F}^{p_j}}=\tilde{c}_{p_j}^{\ast}|_{\mathcal{F}^{p_j}}$ to $\tilde{c}_{p_i}$ where $\tilde{c}_{p_i}|_{\mathcal{B}^{p_i}}\in Q^{p_i,\ast}_F|_{\mathcal{B}^{p_i}}$. Apply $DijksTargets$ method [add ref] to synthesize $R^{\varpi,p}_+$, where all the $\tilde{c}_{p_j}$ are set as source nodes and all $\tilde{c}_{p_i}$ are set as target nodes. Ultimately, $R^{\varpi}_+$ is synthesized:
%$$R^{\varpi}_+ = R^{\varpi,p}_+\oplus R^{\varpi,s}_-$$
Then all the motion paths are synthesized and the candidate with minimal $Cost(R^{\varpi}_+)$ is elected to inherit the task of $p_i$.\par
When $p_i$ is fixed properly, the agent inheriting the task of $p_i$ is driven to return to its own task by reversing above method. The entire procedure of constructing $\widetilde{\mathcal{U}}_{ij}$ and each candidate finding out $R^{\varpi}_+$ is exhibited by Algorithm 4 and $\emph{Problem 2}$ is solved.

\begin{algorithm}
  \caption{the construction of $\widetilde{\mathcal{U}}_{ij}$ and the synthesis of $R^{\varpi}_+$}
  \KwIn{$\mathcal{C}^{p_i}_n$, $\mathcal{C}^{p_j}_n$ and $\tilde{c}_{p_j}^{\ast}$}
  \KwOut{$\widetilde{\mathcal{U}}_{ij}\ and\  R^{\varpi}_+$}

 % $\widetilde{Q}^{ij}=Q^{p_i,\ast}\cup Q^{p_j,\ast}$
%
%  $\tilde{\delta}^{ij}=\delta^{p_i}\cup \delta^{p_j}$
%
%  $\widetilde{Q}^{ij}_0=Q^{p_j,\ast}_0|_{\mathcal{F}^{p_j}}\times Q^{p_i,\ast}_0|_{\mathcal{B}^{p_i}}$
%
%  $\widetilde{Q}^{ij}_F=Q^{p_i,\ast}_F$
%
%  $\widetilde{W}^{ij}=W^{p_i,\ast}\cup W^{p_j,\ast}$
%
%  $\tilde{\tau}^{ij}=\tau^{p_i}\cup \tau^{p_j}$

  $\Omega_{p_j}=\emptyset$

  \ForEach{$\tilde{c}_{p_j}\in \mathcal{C}^{p_j}_n$}
  {

    \If{$\tilde{c}_{p_j}|_{\mathcal{F}^{p_j}}=\tilde{c}_{p_j}^{\ast}|_{\mathcal{F}^{p_j}}$}
    {
        add $\tilde{c}_{p_j}$ to $\Omega_{p_j}$
    }
    \ForEach{$\tilde{c}_{p_i}\in \mathcal{C}^{p_i}_n$}
    {
        \If{$\tilde{c}_{p_i}|_{\mathcal{F}^{p_i}}\ is\ bordered\ to\ \tilde{c}_{p_j}|_{\mathcal{F}^{p_j}}$}
        {
            \If{$(\tilde{c}_{p_i},\tilde{c}_{p_j})$ is a couple-edge}
            {
                $\tilde{c}_{p_j}\in \tilde{\delta}^{ij}(\tilde{c}_{p_i})$

                $\widetilde{W}^{ij}(\tilde{c}_{p_i},\tilde{c}_{p_j})=dist(\tilde{c}_{p_i}|_{\mathcal{F}^{p_i}},\tilde{c}_{p_j}|_{\mathcal{F}^{p_j}})$

                $\tilde{\tau}^{ij}(\tilde{c}_{p_i},\tilde{c}_{p_j}) = None$
            }
            \Else
            {
                \If{$L^{p_i}(\tilde{c}_{p_j}|_{\mathcal{F}^{p_i}})\in sym(\varphi_c^{p_j})$}
                {
                    $\tilde{c}_{p_j}\in \tilde{\delta}^{ij}(\tilde{c}_{p_i})$

                    $\widetilde{W}^{ij}(\tilde{c}_{p_i},\tilde{c}_{p_j})=dist(\tilde{c}_{p_i}|_{\mathcal{F}^{p_i}},\tilde{c}_{p_j}|_{\mathcal{F}^{p_j}})$

                    $\tilde{\tau}^{ij}(\tilde{c}_{p_i},\tilde{c}_{p_j})=L^{p_i}(\tilde{c}_{p_i}|_{\mathcal{F}^{p_i}})$
                }
            }
        }
    }
  }
  $R^{\varpi}_+=DijksTA(\Omega_{p_j},\widetilde{Q}^{ij}_F)$


  \Return{$\widetilde{\mathcal{U}}_{ij}\ and\  R^{\varpi}_+$}
\end{algorithm}




\section{Overall Structure}
Given the task specification $\varphi^{p_i}$ and FTS $\mathcal{F}^{p_i}$ of each agent, decouple the tight coupled task specification and construct the new product automaton $\mathcal{C}^{p_i}_n$ through applying Algorithm 1. The initial motion path $\varrho^{p_i}$ of each agent $p_i$ can be synthesized. To guarantee the collaboration, an online collaboration scheme in \cite{guo2017task} is employed. In the scenarios that the decentralized motion planning is invalid such as the task inheriting and swapping, the union agent model is proposed to dispatch agents to strive to complete the task when part of agents failure and reduce the computational complexity simultaneously. The task inheriting is triggered by specific conditions, such as $\textbf{Warning}^{p_i}$ messages are transmitted through the communication network while the online collaboration system is running.
\section{Case Study}
Consider the logistics system in a depot, three groups of agents are assigned different task specifications. For simplicity, denote the agents by $g_k={p_{i,k},i=1,2,...},\forall k=1,2,3$. The algorithms proposed are implemented and simulated on a laptop (i5-7300HQ, 2.50GHz Duo CPU and 8GB RAM).
\subsection{Workspace Description}
The environment of depot is shown in Fig.\ref{envir} and the initial positions of agents are shown in Fig.\ref{initial}. The blue strips are the conveyor belts in the depot and agents cannot across them. The regions labeled as $ti$ and $mi$ ($i=1,2,...,6$) are checkpoints around the conveyor belts and the agents working on belts need to visit them frequently. Two groups of agents which are silver in Fig.\ref{initial} are responsible for the work on the conveyor belts and need to take care of the switches which are labeled by $o$ and $o'$. The regions known by these two groups are painted yellow and green. A group of agents which are red in Fig.\ref{initial} are assigned to collect goods in $g$, move across the door $d$ and place the goods at the cargo collection point which is labeled $r$. The regions known by them are painted gray.\par
The switches in gray control the door $d$ and when the switch is open, it will change to yellow. Only when $o$ and $o'$ are open simultaneously, the door is open and agents can move across it. An agents will change to blue when it sends out request messages. Meanwhile, when an agent replies to the request message, it will change to black and there will be a red line connecting them. In Fig.\ref{initial}, the agent in $r$ has sent a request message because it intends to move across the door $d$, and two agents reply to the request message and turn to black. Two red lines connect each of them to a request agent.
\begin{figure}
\centering
\subfigure[environment]{
\label{envir}
\includegraphics[width=1.2in]{simulate1/envir.png}}
\hspace{0.3in}
\subfigure[initial positions]{
\label{initial}
\includegraphics[width=1.2in]{simulate1/initial.png}}
\caption{(a) shows the simulation environment, where the relevant regions of different groups are in different colors and some pivotal regions are labeled. (b) shows the initial positions of agents.}
\label{first}
\end{figure}
\subsection{Task Specification}
Three different tasks are assigned to agents. The tasks of agents working on conveyor belts are specified as ($\square\Diamond t1 \wedge \square\Diamond t2 \wedge \square\Diamond t3 \wedge \square\Diamond t4 \wedge \square\Diamond t5 \wedge \square\Diamond t6 \wedge \square(\neg b)$) and ($\square\Diamond m1 \wedge \square\Diamond m2 \wedge \square\Diamond m3 \wedge \square\Diamond m4 \wedge \square\Diamond m5 \wedge \square\Diamond m6 \wedge \square(\neg b)$) which means that they need to visit all the checkpoints frequently and never move across the conveyor belts. The task of agents which need to collect goods and place them in cargo collection point is specified as ($\square\Diamond g \wedge \square(g\Longrightarrow \Diamond r) \wedge \square(\neg b) \wedge \square(\neg d \cup (o \wedge o'))$). The goods collection agents can never move across the belts and only when the switches are both open can they move across the door $d$. $\square(\neg r \cup (o \wedge o'))$ is the tight coupled task $\varphi_c$.
\subsection{Simulation Result}
The motion paths of agents are shown in Fig.\ref{path}, where different colors indicate different states of agents. The agents which are responsible for delivering  goods are denoted by $g_1$, they will change from red to blue when they send request messages. Meanwhile, two agents are elected from $g_2$ and $g_3$ to reply the request messages and change from blue to black. Record the position of agents in each time step and synthesize these motion paths of agents. Two typical scenarios are exhibited in Fig.\ref{before} and Fig.\ref{across}.\par
%In Fig.\ref{before}, two agents, denoted by $p_0$ and $p_1$, in $g_1$ are passing through the door $d$ to place the goods in good collection point $r$. At the same time, two agents, denoted by $p_2$ and $p_3$, in $g_2$ and $g_3$ are elected to assist $p_0$ and $p_1$. $p_0$ and $p_1$ are turned to blue, and $p_2$, $p_3$ are turned to black. There are red solid lines connecting $p_0$ with $p_2$ and $p_3$, $p_1$ with $p_2$ and $p_3$, which indicates they are transmitting message.\par
%In Fig.\ref{across}, $p_0$ and $p_1$ have placed the goods in $r$ and they are going back to $g$ to collect goods. $p_0$ has passed through $d$ so it stops the sending of request messages, but $p_1$ is still passing through $d$ so it continues to send the request messages and $p_2$, $p_3$ continue to reply to it.\par
\begin{figure}
\centering
\subfigure[]{
\label{path}
\includegraphics[width=1in]{simulate1/path.png}}
\hspace{0.02cm}
\subfigure[]{
\label{before}
\includegraphics[width=1in]{simulate1/before.png}}
\hspace{0.02cm}
\subfigure[]{
\label{across}
\includegraphics[width=1in]{simulate1/across.png}}
\caption{(a) shows the motion paths of agents, (b) shows the message transmission when two agents are passing through the door and moving to the goods collection point and (c) shows the message transmission when two agents are going back to collect goods after they placed the goods.}
\label{second}
\end{figure}
In order to represent the collaboration of agents more intuitively, the Gantt chart is drawn as Fig.\ref{Gantt}. Each horizontal line in Gantt chart demonstrates the change of agent states over time. If a section of the horizontal line indicates the agent is sending request messages or replying for the request messages of other agents, which is represented by blue or black, the indices of objective agent are identified at the beginning of this section.
\begin{figure}[h]	
\centering  	
\includegraphics[width=1\linewidth]{simulate1/gante_no_grid.png}
\caption{The gantt chart of agents. Different colors indicate different states of agents. The number at the beginning of the blue section is the index of agent sending request messages and the number $p_s$ at the beginning of the black section in the horizontal line of agent $p_t$ indicates the $p_t$ is replying to $p_s$.}
\label{Gantt}
\end{figure}

%\subsection{Prevention Of Long-time Demission}
%In this section we increase the amount of agents in $g_1$ to verify the prevention of long-time demission. Fig.\ref{path2} shows the motion paths of agents in simulation 2 and a typical communication scenario during the simulation is presented in Fig.\ref{message}. The Gantt chart of agents in simulation 2 is shown in Fig.\ref{Gantt2}. From Fig.\ref{message} we can see when the replying agents have to return to its own tasks, they will send a succession signal to other agents belonging to the same group and then succession agents are elected to succeed original replying agents. There are yellow solid lines connecting original replying and succession agents and the message transmissions are transferred to the request agents and new replying agents.
%\begin{figure}
%\centering
%\subfigure[motion paths]{
%\label{path2}
%\includegraphics[width=1.2in]{simulate2/path.png}}
%\hspace{0.3in}
%\subfigure[succession signal]{
%\label{message}
%\includegraphics[width=1.2in]{simulate2/message.png}}
%\caption{(a) shows the motion paths of agents in simulation 2. (b) shows a communication scenario when the original replying agents have to return to its own tasks.}
%\label{second}
%\end{figure}
%
%\begin{figure}[h]	
%\centering  	
%\includegraphics[width=1\linewidth]{simulate2/gante_no_grid.png}
%\caption{The Gantt chart of agents in simulation 2. The yellow parts indicates the transmission of succession signal.}
%\label{Gantt2}
%\end{figure}
\subsection{Task Inheriting}
Assume that the goods collection task is more important than checking the belts. When one goods collection agent is broken, the agents working on belts need to inherit the goods collection task. Fig.\ref{third} shows two motion paths of agents which start from different positions to inherit the goods collection task, where the invariant suffix is denoted by red arrows and the prefix is denoted by blue arrows.
\begin{figure}
\centering
\subfigure[]{
\label{path3}
\includegraphics[width=1.2in]{simulate3/pathonecolor.png}}
\hspace{0.3in}
\subfigure[]{
\label{path4}
\includegraphics[width=1.2in]{simulate3/pathtwocolor.png}}
\caption{(a) and (b) show the motion paths of agents which start from different positions to inherit the goods collection task. The blue arrows represent the path through which the agent proceeds to inherit the task, and red arrows represent the invariant suffix of the accepting run of the inherited task. }
\label{third}
\end{figure}

%\begin{figure}[h]	
%\centering  	
%\includegraphics[width=1\linewidth]{simulate3/gante_no_grid.png}
%\caption{The Gantt chart of agents in simulation 3, where the dashed is placed to show the yellow part and green part are aligned, which indicates when the good collection agent broken down, one agent working on belts went to inherit its task immediately.}
%\label{Gantt3}
%\end{figure}

\subsection{Computational Complexity}
In the first simulation, the team has 14 agents and the simulation environment consists of 108 states and 97 edges, the BA associated with $g_1$ consists of 8 states and 30 transitions, and the BA associated with $g_2$ and $g_3$ consists of 7 states and 34 transitions. As a result, the centralized product automaton will have $108^{14}\times 8^6 \times 7^6(9\times 10^{38})$ states. In our decentralized method, each agent only knows partial environment as shown in Fig.\ref{envir}. The local new product automata of $g_1$, $g_2$ and $g_3$ consist of 136, 266 and 210 states, respectively. It costs an average of 0.02s to synthesize the initial motion path for each agent. Adjust the number of agents and the motion path is synthesized within the same time, which indicates the computational complexity of our decentralized method only associated with the states number of local FTS and BA.\par
In the second simulation, the number of the state number of the union agent model is 402 and the maximum number will less than 476. It costs an average of 0.035s to synthesize the motion path for the agent to inherit the task.
% if have a single appendix:
%\appendix[Proof of the Zonklar Equations]
% or
%\appendix  % for no appendix heading
% do not use \section anymore after \appendix, only \section*
% is possibly needed

% use appendices with more than one appendix
% then use \section to start each appendix
% you must declare a \section before using any
% \subsection or using \label (\appendices by itself
% starts a section numbered zero.)
%

\section{Summary And Future Work}
%\appendices
%\section{Proof of the First Zonklar Equation}
%Appendix one text goes here.

%% you can choose not to have a title for an appendix
%% if you want by leaving the argument blank
%\section{}
%Appendix two text goes here.
The simulation results demonstrate that proposed method can handle the decentralized collaboration of multi-agent system under tight coupled task specifications. The proposed decentralized method depends on local FTS and BA of agent so the computational complexity doesn't increase accompanied by the increasing of agent number, which reduces the computational complexity significantly compared to the centralized method. In the future, we will continue to find methods to further reduce the computational complexity and combine the proposed method with existing control algorithms to improve the overall convergence rate of cyber-physical systems.


% use section* for acknowledgment
%\section*{Acknowledgment}
%
%
%The authors would like to thank...
%
%
%% Can use something like this to put references on a page
%% by themselves when using endfloat and the captionsoff option.
%\ifCLASSOPTIONcaptionsoff
%  \newpage
%\fi



% trigger a \newpage just before the given reference
% number - used to balance the columns on the last page
% adjust value as needed - may need to be readjusted if
% the document is modified later
%\IEEEtriggeratref{8}
% The "triggered" command can be changed if desired:
%\IEEEtriggercmd{\enlargethispage{-5in}}

% references section

% can use a bibliography generated by BibTeX as a .bbl file
% BibTeX documentation can be easily obtained at:
% http://mirror.ctan.org/biblio/bibtex/contrib/doc/
% The IEEEtran BibTeX style support page is at:
% http://www.michaelshell.org/tex/ieeetran/bibtex/
%\bibliographystyle{IEEEtran}
% argument is your BibTeX string definitions and bibliography database(s)
%\bibliography{IEEEabrv,../bib/paper}
%
% <OR> manually copy in the resultant .bbl file
% set second argument of \begin to the number of references


%\begin{thebibliography}{1}
%
%\bibitem{IEEEhowto:kop}
%H.~Kopka and P.~W. Daly, \emph{A Guide to \LaTeX}, 3rd~ed.\hskip 1em plus
%  0.5em minus 0.4em\relax Harlow, England: Addison-Wesley, 1999.
%
%\bibitem{IEEEhowto}
%GASTIN, Paul, ODDOUX, et al, \emph{Fast LTL to B\"{u}chi automata translation},\hskip 1em plus
%  0.5em minus 0.4em\relax International Conference on Computer Aided Verification, 2001.
%
%\bibitem{IEEEhowto:k}
%H.~Kopka and P.~W. Daly, \emph{A Guide to \LaTeX}, 3rd~ed.\hskip 1em plus
%  0.5em minus 0.4em\relax Harlow, England: Addison-Wesley, 1999.
%
%\end{thebibliography}





% biography section
%
% If you have an EPS/PDF photo (graphicx package needed) extra braces are
% needed around the contents of the optional argument to biography to prevent
% the LaTeX parser from getting confused when it sees the complicated
% \includegraphics command within an optional argument. (You could create
% your own custom macro containing the \includegraphics command to make things
% simpler here.)
%\begin{IEEEbiography}[{\includegraphics[width=1in,height=1.25in,clip,keepaspectratio]{mshell}}]{Michael Shell}
% or if you just want to reserve a space for a photo:

%\begin{IEEEbiography}{Michael Shell}
%Biography text here.
%\end{IEEEbiography}
%
%% if you will not have a photo at all:
%\begin{IEEEbiographynophoto}{John Doe}
%Biography text here.
%\end{IEEEbiographynophoto}
%
%% insert where needed to balance the two columns on the last page with
%% biographies
%%\newpage
%
%\begin{IEEEbiographynophoto}{Jane Doe}
%Biography text here.
%\end{IEEEbiographynophoto}

% You can push biographies down or up by placing
% a \vfill before or after them. The appropriate
% use of \vfill depends on what kind of text is
% on the last page and whether or not the columns
% are being equalized.

%\vfill

% Can be used to pull up biographies so that the bottom of the last one
% is flush with the other column.
%\enlargethispage{-5in}

% that's all folks
\bibliography{mybib}
\end{document}


